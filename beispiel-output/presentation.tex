% Created 2015-12-10 Do 22:49
\documentclass[t,compress,11pt,xcolor=dvipsnames]{beamer}
\usepackage[utf8]{inputenc}
\usepackage[T1]{fontenc}
\usepackage{fixltx2e}
\usepackage{graphicx}
\usepackage{longtable}
\usepackage{float}
\usepackage{wrapfig}
\usepackage{rotating}
\usepackage[normalem]{ulem}
\usepackage{amsmath}
\usepackage{textcomp}
\usepackage{marvosym}
\usepackage{wasysym}
\usepackage{amssymb}
\usepackage{hyperref}
\tolerance=1000
\usepackage[ngerman]{babel}
\usepackage{url}
\usepackage{pdfpcnotes}
\usepackage[bars]{beamerthemetree} %Beamer theme v 2.2
\usepackage{kerkis}
\usepackage{multimedia}

\definecolor{HMRed}{RGB}{211, 3, 48}
\usecolortheme[named=HMRed]{structure}
\usefonttheme[onlymath]{serif}
\setbeamercovered{highly dynamic}
\usetheme{Ilmenau} % Beamer theme v 3.0
\useoutertheme[subsection=true]{smoothbars}%Beamer Outer Theme-circles on top
\useinnertheme{circles} %rectangle bullet points instead of circle ones

% beamer stuff (font sizes and colors)
\setbeamerfont{section in toc}{size=\small}
\setbeamerfont{subsection in toc}{size=\footnotesize}
\setbeamerfont{footnote}{size=\tiny}
\setbeamerfont{caption}{size=\tiny}
\setlength\abovecaptionskip{-5pt}
\setbeamerfont{alerted text}{series=\bfseries}
\setbeamercolor{alerted text}{fg=HMRed}

% hide navigation bar
\usenavigationsymbolstemplate{}
% current frame number in ilmenau
\newcommand*\oldmacro{}%
\let\oldmacro\insertshorttitle%
\renewcommand*\insertshorttitle{%
    \oldmacro\hfill%
\insertframenumber\,}%/\,\inserttotalframenumber

% footnotes in brackets
% distinguish between exponents
\renewcommand*{\thefootnote}{(\arabic{footnote})}

% institution title and logo
\institute{Hochschule München}
\logo{%
    \makebox[0.95\paperwidth][r]{%
        \includegraphics[width=1cm,height=1cm,keepaspectratio]{img/hm_edu.jpg}~%
    }%
}

% toc for every section
% title page for every section
\AtBeginSection[]{
    % Looks nice, but not very helpfull
    % \begin{frame}
    % \vfill
    % \centering
    % \begin{beamercolorbox}[sep=8pt,center,rounded=true]{title}
    %   \usebeamerfont{title}\insertsectionhead\par%
    % \end{beamercolorbox}
    % \vfill
    % \end{frame}
    \begin{frame}
        \frametitle{\insertsectionhead}
        \tableofcontents[currentsection]
    \end{frame}
}

\usetheme{default}
\author{Armin Grodon}
\date{10.12.2015}
\title{Präsentationstemplate\\ mit Zeilenumbruch im Titel}
\hypersetup{
  pdfkeywords={LaTeX beamer presentation MUAS HM munich},
  pdfsubject={LaTeX-Template für die Hochschule München},
  pdfcreator={Emacs 24.5.1 (Org mode 8.2.10)}}
\begin{document}

\maketitle
\begin{frame}{Gliederung}
\tableofcontents
\end{frame}



\section{Erste Section}
\label{sec-1}
\subsection{Erster Unterpunkt}
\label{sec-1-1}
\begin{frame}[label=sec-1-1-1]{Slide 1}
\begin{columns}
\begin{column}{0.6\textwidth}
Slide mit Text und einem Bild
\begin{itemize}
\item Fußnoten in Slides mit mehreren Columns zu bekommen
ist etwas hässlich\footnotemark[1]
\item Man hat sie entweder in der Column in der sie benutzt werden
\begin{itemize}
\item oder muss sie selbst platzieren
\item in einer eigenen Column
\end{itemize}
\item \emph{Alternativ verzichtet man auf Fußnoten in solchen Layouts}
\item Aber man sollte sich für eine Art von Fußnoten entscheiden
\end{itemize}
\pnote{Notes für pdfpc}
\pnote{Keine Zeilenumbrüche in pnote-Element!}
\end{column}
\begin{column}{0.4\textwidth}
\begin{figure}[htb]
\centering
\includegraphics[width=0.8\textwidth]{./img/cat.jpg}
\caption{Eine Katze\footnotemark[2]!}
\end{figure}
\end{column}
\end{columns}
\footnotetext[1]{Persönlicher Eindruck vom benötigten inline-LaTeX-Code}
\footnotetext[2]{\url{https://commons.wikimedia.org/wiki/File:So_happy_smiling_cat.jpg}}
\end{frame}
\subsection{Zweiter Unterpunkt}
\label{sec-1-2}
\begin{frame}[label=sec-1-2-1]{Slide 2}
Slides mit nur Text sind dafür \alert{sehr} übersichtlich

\begin{center}
\begin{tabular}{l|lcl|lr|}
 & A & B &  & C & D\\
\hline
chicken\footnotemark & Q & Q &  &  & \\
 & QQQQQQ & Q &  & x & xxxx\\
 &  & QQ &  &  & x\\
 &  & QQQ &  & x & xx\\
\hline
chicken chicken & Q & QQQQ &  & x & x\\
 &  &  &  &  & x\\
\end{tabular}
\end{center}\footnotetext[1]{Tabellen auch}
\end{frame}

\begin{frame}[fragile,label=sec-1-2-2]{Slide 3}
 \begin{block}{Column 1}
Text in einer Column\footnote{Footnote in einer Column}
\end{block}
\begin{block}{Column 2}
\begin{verbatim}
a^2 * b^2 = c^2
\end{verbatim}
\end{block}
\end{frame}
\subsection{Dritter Unterpunkt}
\label{sec-1-3}
\begin{frame}[label=sec-1-3-1]{Slide 4}
Org-Mode-Fußnoten stimmen auch nicht mehr,
wenn man zusätzlich Inline-\LaTeX{}-Fußnoten verwendet\footnote{Keine Ahnung, wie man das lösen kann}
\end{frame}
\begin{frame}[label=sec-1-3-2]{Slide 5}
Columns können auch ohne Überschrift und Rahmen verwendet werden\\
     "`Text in Quotes"'\footnote{Footnote mit Name}
\begin{itemize}
\item A\footnote{Anonyme Fußnote}
\item B\footnote{Inline Fußnote}
\end{itemize}
\footnotetext[1234]{Fußnote an total falscher Stelle\footnotemark[10]}
\footnotetext[10]{Von Hand gesetzte Fußnoten sind richtig nervig zu verwalten}
\end{frame}
\section{Zweite Section}
\label{sec-2}
\subsection{Unterpunkt}
\label{sec-2-1}
\begin{frame}[label=sec-2-1-1]{}
\begin{itemize}
\item Seitenzahlen auf \emph{appendix}-Slides entfernen
\item Quellen schöner machen
\item Fußnotenproblem lösen
\item Weniger Zeit mit \LaTeX"" verschwenden
\end{itemize}
\end{frame}
\section{Letze Section}
\label{sec-3}
\subsection{Letzer Unterpunkt}
\label{sec-3-1}
\begin{frame}[label=sec-3-1-1]{Letzte Slide}
Alles danach ist nicht mehr im TOC
\end{frame}
\appendix
Quellen:
\tiny\begin{itemize}
\item \url{https://commons.wikimedia.org/wiki/File:So_happy_smiling_cat.jpg}
\item The Internet
\item
\item
\item
\end{itemize}
\small\centering Vielen Dank für die Aufmerksamkeit
% Emacs 24.5.1 (Org mode 8.2.10)
\end{document}
